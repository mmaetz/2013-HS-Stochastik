\chapter{Stetige Wahrscheinlichkeitsverteilung}
Bei einer stetigen Zufallsvariablen $X$ ist $\p(X=x)=0$ für jedes feste $x$. Wir betrachten nur Fälle, wo $\p(x\leq X\leq x+h)$ für kleine $h$ ungefähr proportional zu $h$ ist. Die Proportionalitätskonstante heisst die \emph{Dichte} $f$ von $X$.
%
\section{Wahrscheinlichkeitsdichte}
Die Dichte von einer stetigen Verteilung $P$ ist definiert als
\begin{align*}
	f(x)&=\lim_{h\downarrow 0}\frac{\p(x\leq X\leq x+h)}{h}
	\intertext{Zwischen der Dichte $f$ und der kumulativen Verteilungsfunktion $F$ bestehen die folgenden Beziehungen:}
	f(x) &= F'(x), \ \ F(x) = \int_{-\infty}^x\!\dd{x}\, f(x)
	\intertext{Erwartungswert und Varianz berechnen sich gemäss}
	\E(X) &= \mu_X = \int_{-\infty}^\infty\!\dd{x}\, xf(x) \\
	\V(X) &= \sigma^2_X = \int_{-\infty}^\infty\!\dd{x}\, (x-\mu_X)^2f(x)
	\intertext{und es gelten die gleichen Rechenregeln wie im diskreten Fall (TODO: Referenz hier?).}
\end{align*}
%
\section{Die wichtigsten stetigen Verteilungen}
\subsection{Uniforme Verteilung}
Die \emph{uniforme Verteilung} $\uni[a,b]$ tritt auf bei Rundungsfehlern und als Formalisierung der völligen "Ignoranz".
\begin{align*}
	W&=[a,b]\\
	f(x)&=\frac{1}{b-a}1_{[a,b]}(x)=
	\begin{cases}
		\frac{1}{b-a} & a\leq x \leq b \\
		0 & \text{sonst}
	\end{cases}\\
	\E(X)&=\frac{a+b}{2}\\
	\sigma_X&=\frac{b-a}{\sqrt{12}}
\end{align*}
%
\subsection{Exponentialverteilung}
Die \emph{Exponentialverteilung} $\exp(\lambda)$ ist das einfachste Modell für Wartezeiten auf Ausfälle und eine stetige Version der geometrischen Verteilung.
\begin{align*}
	W&=[0,\infty)\\
	f(x)&=\lambda e^{-\lambda x}, \qquad \text{für}\ x>0\\
	F(x)&=1-e^{-\lambda x}\\
	\E(X)&=\sigma_X=\frac{1}{\lambda}
\end{align*}
Wenn die Zeiten zwischen den Ausfällen eines Systems Exponential$(\lambda)$-verteilt sind, dann ist die Anzahl Ausfälle in einem Intervall der Länge $t$ $\text{Poi}(\lambda t)$-verteilt.
%
\subsection{Normal- oder Gaussverteilung}
Die \emph{Normal- oder Gaussverteilung} $\mathcal{N}(\mu,\sigma^2)$ ist die häufigste Verteilung für Messwerte.
\begin{align*}
	W&=\mathbb{R}\\
	f(x)&=\frac{1}{\sqrt{2\pi}\sigma} \exp\left(-\frac{1}{2}\left(\frac{x-\mu}{\sigma}\right)^2\right)\\
	\E(X)&=\mu\\
	\sigma_X&=\sigma
\end{align*}
Die $\mathcal{N}(0,1)$Verteilung, auch \emph{Standardverteilung} bezeichnet, ist ein wichtiger Sonderfall, weshalb es für dessen Verteilungsfunktion sogar ein eigenes Symbol gibt: $\Phi(x) := F_{\mathcal{N}(0,1)}, \ x\in\mathbb{R}$ und deren Umkehrfunktionen (d.h. die Quantile) kürzen wir ab mit $z_\alpha := \Phi^{-1}(\alpha), \ \alpha \in [0,1]$. Die Verteilungsfunktion $F$ einer $\mathcal{N}(\mu,\sigma^2)$-verteilten Zufallsvariable ist nicht geschlossen darstellbar, sie wird aus der Verteilungsfunktion $\Phi$ der Standardnormalverteilung (welche tabelliert ist) berechnet mittels der Formel:
\begin{equation}
	F(x)=\Phi\left(\frac{x-\mu}{\sigma}\right), \ \ x \in \mathbb{R}
\end{equation}
%
\section{Transformationen}
Bei stetigen Verteilungen spielen Transformationen $Y=g(X)$ eine wichtige Rolle. Falls $g$ \emph{linear} ist: $g(x)=a+bx$ mit $b>0$, dann gilt $\E(Y)=a+b\E(X)$, $\sigma_Y=b\sigma_X$, $F_Y(x)=F_X((x-a)/b)$ und $f_Y(x)=f_X((x-a)/b)/b$. Durch Skalenänderungen kann man also alle Exponentialverteilungen ineinander überführen, und ebenso durch lineare Transformationen alle Normalverteilungen ineinander. Für beliebiges $g$ gilt
\begin{equation}
	\E(Y)=\E(g(X))=\int_{-\infty}^\infty\!\rd x\, g(x)f(x)
\end{equation}
Wenn $X\sim \mathcal{N}(\mu,\sigma^2)$ normalverteilt ist, dann heisst $Y=e^X$ \emph{lognormal-verteilt}. Es gilt z.B. $\E(Y)=\exp(\mu+\sigma^2/2)$.
%
\section{Simulation von Zufallsvariablen}
Wenn $U$ uniform auf $[0,1]$ verteilt ist und $F$ eine beliebige kumulative Verteilungsfunktion, dann ist die Verteilungsfunktion von $X=F^{-1}(U)$ gleich $F$. Dies ist ein wichtiges Faktum um Verteilungen, respektive Realisierungen von Zufallsvariabeln, zu \emph{simulieren}:
\begin{compactenum}[1.]
	\item Erzeuge Realisation $u$ von uniform verteilter Zufallsvariable $U\sim\uni[0,1]$. Dies wird mittels einem "Standard-Paket" gemacht.
	\item Berechne $x=F^{-1}(u)$. Gemäss obigem Faktum ist dann $x$ eine Realisation einer Zufallsvariablen $X$ mit kumulativer Verteilungsfunktion $F$.
\end{compactenum}
Diese Methode ist nicht immer rechentechnisch effizient.
