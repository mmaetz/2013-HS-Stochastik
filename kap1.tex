\chapter{Der Begriff der Wahrscheinlichkeit}
\label{kap1}
Stochastik befasst sich mit \emph{Zufallsexperimenten}. Deren Ergebnisse sind unter \glqq Versuchsbedingungen\grqq verschieden.
\begin{bspl}
	\begin{compactitem}
		\item Kartenziehen, Würfeln, Roulette
		\item Simulation
		\item Komplexe Phänomene (zumindest approximativ): Börse, Data-Mining, Genetik, Wetter
	\end{compactitem}
\end{bspl}
Ergebnisse von Zufallsexperimenten werden in \emph{Ereignisse} zusammengefasst.
\begin{compactitem}
	\item \emph{Ereignisraum} (Grundraum) $\Omega$: Menge aller möglichen Ergebnisse des Zufallsexperiments
	\item \emph{Elementarereignisse $\omega$}: Elemente von $\Omega$, also die möglichen Ergebnisse des Zufallsexperiments
	\item \emph{Ereignis}: Teilmenge von $\Omega$
	\item Operationen der Mengenlehre haben natürliche Interpretation in der Sprache der Ereignisse:\\[-1em]
%		\begin{table}
		\begin{center}
				\begin{tabular}{lcl}
					\toprule
					Durchschnitt	& $A\cap B$			& $A$ \emph{und} $B$	\\
					Vereinigung		& $A\cup B$			& $A$ \emph{oder} $B$	\\
					Komplement		& $A^c$				& \emph{Nicht} $A$	\\
					Differenz		& $A\setminus B$	& $A$ \emph{ohne} $B$ \\
					\bottomrule
				\end{tabular}
		\end{center}
		$A,B$ heissen
		\begin{center}
				\begin{tabular}{lr}
					\toprule
					disjunkt	&falls $A\cap B=\varnothing$			\\
					\bottomrule
				\end{tabular}
		\end{center}
%			\label{tab:main}
%		\end{table}
\end{compactitem}

Das Vorgehen der Stochastik zur Lösung eines Problems kann in drei Schritte unterteilt werden:
\begin{compactenum}[1.]
	\item Man bestimmt die Wahrscheinlichkeiten gewisser Ereignisse $A_i$. Dabei sind Expertenwissen, Daten und Plausibilitäten wichtig.
	\item Man berechnet aus den Wahrscheinlichkeiten $\p(B_j)$ die Wahrscheinlichkeiten von gewissen anderen Ereignissen $B_j$ gemäss den Gesetzen der Wahrscheinlichkeitstheorie (oft vereinfachend unter Unabhängigkeitsannahme).
	\item Man interpretiert die Wahrscheinlichkeiten $\p(B_j)$ im Hinblick auf die Problemstellung.
\end{compactenum}
Das \emph{Bestimmen von Wahrscheinlichkeiten} (siehe Schritt 1) wird oft konkreter formalisiert. 
\begin{bspl}[Laplace-Modell]
	Die Wahrscheinlichkeit von einem Ereignis $A$ ist gegeben durch
	\begin{align*}
		\p(A)&=\frac{\abs{A}}{\abs{\Omega}}\quad \left( =\frac{\text{\#günstige Fälle}}{\text{\#mögliche Fälle}} \right).
		\intertext{Dem Laplace-Modell liegt die \emph{uniforme Verteilung} von  Elemntarereignissen $\omega$ zugrunde:}
		\p(\{\omega\})&=\frac{1}{\abs{\Omega}}
		\label{}
	\end{align*}
\end{bspl}
Andere Wahrscheinlichkeitsverteilungen werden mit Hilfe des Konzepts von \emph{Zufallsvariablen} (siehe Kapitel \ref{kap2}) eingeführt. Es sei aber bereits hier festgehalten: die Stochastik geht weit über das Laplace-Modell hinaus. Für viele Anwendungen ist das Laplace-Modell ungeeignet.

\section{Rechenregeln für Wahrscheinlichkeiten}
Die drei Axiome sind:
\begin{compactenum}[({A}1)]
	\item $\p(A)\geq 0$: Wahrscheinlichkeiten sind immer nicht-negativ.
	\item $\p(\Omega)=1$: Sicheres Ereignis $\Omega$ hat Wahrscheinlichkeit eins.
	\item $\p(A\cup B)=\p(A)+\p(B)$ $\forall$ Ereignisse $A, B$, die sich gegenseitig ausschliessen (d.h. $A\cap B=\varnothing$).
\end{compactenum}
Weitere (abgeleitete) Regeln:
\begin{align*}
	\p(A^{c})&=1-\p(A)
	\shortintertext{für jedes Ereignis $A$,}
	\p(A\cup B)&=\p(A)+\p(B)-\p(A\cap B)\\
	\shortintertext{\text{für je zwei Ereignisse $A$ und $B$},}
	\p(A_1\cup \cdots \cup A_n)&\leq \p(A_1)+\cdots + \p(A_n)
	\shortintertext{für je $n$ Ereignisse $A_1,\ldots,A_n$}
	\p(B\setminus A)&=\p(B)-\p(A)
\end{align*}
	für je zwei Ereignisse $A$ und $B$ mit $A\subset B$.

\section{Unabhängigkeit von Ereignissen}
Wenn zwischen zwei Ereignissen $A$ und $B$ kein kausaler Zusammenhang besteht (d.h. es gibt keine gemeinsamen Ursachen oder Ausschliessungen), dann werden sie \emph{unabhängig} genannt, genauer: Zwei Ereignisse $A$ und $B$ heissen (\emph{stochastisch}) \emph{unabhängig} wenn für jedes $k\leq n$ und alle $1\leq i_1< \cdots < i_k \leq n$ gilt
\begin{gather*}
	\p(A_{i_1}\cap \cdots \cap A_{i_{k}})=\p(A_{i_1})\cdot\cdots\cdot \p(A_{i_k}).
\end{gather*}
Achtung: Zwei \emph{unabhängige} Ereignisse $A$ und $B$ sind \emph{nicht disjunkt} (und umgekehrt), vorausgesetzt die Wahrscheinlichkeiten $\p(A), \p(B) \neq 0$.

\section{Interpretation von Wahrscheinlichkeiten}
Die beiden wichtigsten Interpretationen sind:
\begin{itemize}
	\item Frequentistisch: \glqq Idealisierung der relative Häufigkeiten bei vielen unabhängigen Wiederholungen\grqq
	\item Subjektive: \glqq Mass für den Glauben, dass ein Ereignis eintreten wird\grqq (Bayes'sch)
\end{itemize}
